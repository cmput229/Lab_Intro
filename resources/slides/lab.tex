\documentclass{beamer}

\usepackage{graphicx}

\useinnertheme{rectangles}
\useoutertheme{infolines}
\usecolortheme{crane}
\usefonttheme{structurebold}

\title{Welcome to the Lab}
\author[CMPUT 229]{CMPUT 229}
\institute{University of Alberta}
\date{Winter 2018}

\begin{document}
\frame{\titlepage}

\frame{
  \frametitle{Outline}
  \tableofcontents
}

\section{About the Lab}

\frame{
  \frametitle{The Lab}

 

  \begin{itemize}
    \item The lab sessions are mostly time for you to work.  Attendance is not
      mandatory.
  \end{itemize}
}

\frame{
  \frametitle{The Lab}

 

  \begin{itemize}
    \item The lab sessions are mostly time for you to work.  Attendance is not
      mandatory.
    \item Your TAs are available during the lab sessions.  Ask us questions if
      you have them!
  \end{itemize}
}

\frame{
  \frametitle{The Lab}

 

  \begin{itemize}
    \item The lab sessions are mostly time for you to work.  Attendance is not
      mandatory.
    \item Your TAs are available during the lab sessions.  Ask us questions if
      you have them!
    \item There will sometimes be presentations (like this).  We will let you
      know in advance when these are.
  \end{itemize}
}

\frame{
  \frametitle{The Lab}

 

  \begin{itemize}
    \item The lab sessions are mostly time for you to work.  Attendance is not
      mandatory.
    \item Your TAs are available during the lab sessions.  Ask us questions if
      you have them!
    \item There will sometimes be presentations (like this).  We will let you
      know in advance when these are.
    \item You can attend any lab section, but registered students always have
      priority.
  \end{itemize}
}

\frame{
  \frametitle{What Will We Be Doing?}

  \begin{itemize}
    \item In short: assembly language programming for the MIPS architecture.
    \item But writing assembly for real hardware is hard:
      \begin{itemize}
        \item You have to interface with the operating system, or write lots of
          low-level code to manage the hardware yourself.
        \item Testing and debugging requires special equipment and is difficult.
      \end{itemize}
    \item So we use a simulator: SPIM.
  \end{itemize}
}

\section{SPIM and XSPIM}

\frame{
  \frametitle{What is SPIM?}

  \begin{itemize}
    \item SPIM is MIPS backwards!
    \item It's a MIPS simulator.
    \item It gives you an easy environment in which to run and debug your MIPS
      code.
    \item XSPIM is a graphical interface to SPIM.
    \item There's more information in appendix B of the textbook.
  \end{itemize}

  \pause

  Some things to remember:

  \begin{itemize}
    \item SPIM uses the endianness of the host machine.
    \item SPIM is not a cycle-accurate simulator: it doesn't accurately simulate
      cache, memory latency, floating-point operations, etc.  You can't tell how
      efficient your MIPS code is by running it in SPIM.
  \end{itemize}
}

\frame{
  \frametitle{Pseudo-Instructions}

  \begin{itemize}
    \item SPIM implements some pseudo-instructions that expand into real
      instructions.  When you run code, you might see instructions you didn't
      write.  For example:

      \vspace{0.5em}

      \begin{center}
        \begin{tabular}{l|l}
          What you wrote&What you see in SPIM\\
          \hline
          \texttt{li \$t0, 10}&\texttt{ori \$8, \$0, 10}\\
          \texttt{move \$v0, \$t0}&\texttt{addu \$2, \$0, \$8}
        \end{tabular}
      \end{center}

      \vspace{0.5em}

    \item Pseudo-instructions will do what you intend, so they're nothing to
      worry about.  Just be aware that they may expand to multiple instructions,
      and don't be surprised when you see them.
  \end{itemize}
}

\section{Using XSPIM}

\frame{
  \frametitle{Basic XSPIM Usage}

  \begin{itemize}
    \item Run \texttt{xspim} from a terminal.
    \item Press the ``load'' button, type in a filename, and click ``assembly
      file'' to load a file.
    \item Press the ``run'' button and then click ``ok'' to run your program.
    \item If your program needs input or prints output, XSPIM will pop up a
      terminal.  {\bf Closing this terminal will kill XSPIM.}
    \item Press the ``quit'' button to quit.
  \end{itemize}
}

\section{Using SPIM}

\frame{
  \frametitle{Basic SPIM Usage}

  \begin{itemize}
    \item Start \texttt{spim} in a terminal.
    \item Type \texttt{load ``file.s''} to load your program.
    \item Type \texttt{run} to run your program.
    \item If your program needs input or prints output, it will appear in the
      terminal.
    \item Type \texttt{exit} to quit.
      
      \vspace{0.5em}
    \item You can always type \texttt{?} to get help!
  \end{itemize}
}

\section{Assignment Tips}

\frame{
  \frametitle{Assignment Tips for CMPUT 229}

  \begin{itemize}
    \item Read specifications very carefully.  Pay special attention to what you
      are to include - sometimes we don't want a \texttt{main} method.
    \item Even if you don't finish an assignment, make sure you submit something
      that loads and runs.
    \item Test your assignments on the lab machines before you submit.  That's
      where we'll be marking them.
    \item For most assignments, the marksheet will be posted in advance.  Look
      at it to get an idea of what's important.
    \item Style marks are easy marks.  Format your code like the
      \texttt{example.s} file we provide, and write good comments.
    \item With moodle (eClass) you can submit as many times as you like.  Submit early and
      often (e.g. every time you get part of the assignment working).  We don't
      accept any late submissions!
  \end{itemize}
}

\section{Questions?}

\frame{
  \frametitle{Questions?}
}

\end{document}
